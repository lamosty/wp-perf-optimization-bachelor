\newglossaryentry{caching}
{
	name={cache},
	description={a component that stores data so future requests for that data can be served faster; the data stored in a cache might be the results of an earlier computation \cite{Wiki:Cache}}
}

\newglossaryentry{dummy data}
{
	name={dummy data},
	description={information that does not contain any useful data, but serves to reserve space where real data is nominally present; can be used as a placeholder for both testing and operational purposes \cite{Wiki:Dummy-data}}
}

\newglossaryentry{server resources}
{
	name={server resources},
	description={the amount of CPU and RAM your Web site can use on the server that you are on \cite{WebHostingShow:Server-resources}}
}

\newglossaryentry{opcode}
{
	name={opcode},
	description={the portion of a machine language instruction that specifies the operation to be performed; beside the opcode itself, instructions usually specify the data they will process, in form of operands \cite{Wiki:Opcode}}
}


\newglossaryentry{load testing}
{
	name={load testing},
	description={the process of putting demand on a system or device and measuring its response; load testing is performed to determine a system’s behavior under both normal and anticipated peak load conditions \cite{Wiki:Load-testing}}
}

\newglossaryentry{server stack}
{
	name={server stack},
	description={a set of software subsystems or components needed to create a complete platform such that no additional software is needed to support applications; applications are said to "run on" or "run on top of" the resulting platform. \cite{Wiki:Server-stack}}
}
